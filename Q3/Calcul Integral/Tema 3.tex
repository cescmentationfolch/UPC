\documentclass[12pt]{article}
\usepackage{geometry}
\usepackage[dvipsnames]{xcolor}
\usepackage{fancyhdr}
\usepackage{amsmath , amsthm , amssymb }
\usepackage{graphicx}
\usepackage{hyperref}
\usepackage[utf8]{inputenc}
\title{Calcul Integral tema 3}

\begin{document}
\maketitle
\section{Longitud d'una Corba}
\subsection{Cam\'i o Corba parametritzada}


\label{Definicio 1}
\textbf{\color{ForestGreen} Definici\'o} : Aplicacio \textbf{\color{blue} cont\'inua}
\begin{equation*}
 \gamma : I -> \mathbb{R}^n, \quad I \hspace{4pt} interval \hspace{4pt} no  \hspace{4pt} degenerat
\end{equation*}
C = $\gamma$(I) s'anomena Corba, $\gamma$ es una parametritzacio de C.

\subsection{Longitud} 
\label{Definicio 2}
\textbf{\color{ForestGreen} Definici\'o} : P partici\'o, $\gamma$ parametritzaci\'o \\
\begin{equation*}
L(\gamma,P) = \sum_i^n ||\gamma(t_i) - \gamma(t_{i-1})||
\end{equation*}
\label{Definicio 3}
\textbf{\color{ForestGreen} Definici\'o} : La longitud d'una Corba \'es:
\begin{equation*}
L(\gamma) = \sup_P L(\gamma,P)
\end{equation*}
\label{

\end{document}