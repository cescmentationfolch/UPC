% Created 2018-02-06 Tue 23:46
% Intended LaTeX compiler: pdflatex
\documentclass[11pt]{article}
\usepackage[utf8]{inputenc}
\usepackage[T1]{fontenc}
\usepackage{graphicx}
\usepackage{grffile}
\usepackage{longtable}
\usepackage{wrapfig}
\usepackage{rotating}
\usepackage[normalem]{ulem}
\usepackage{amsmath}
\usepackage{textcomp}
\usepackage{amssymb}
\usepackage{capt-of}
\usepackage{hyperref}
\usepackage[margin=3cm]{geometry}
\usepackage{xfrac}
\date{\today}
\title{Examen Quimica}
\hypersetup{
 pdfauthor={},
 pdftitle={Examen Quimica},
 pdfkeywords={},
 pdfsubject={},
 pdfcreator={Emacs 25.3.1 (Org mode 9.1.3)}, 
 pdflang={English}}
\begin{document}

\maketitle
\section{[2.5pt]}
\label{sec:org220dcd1}
En el laboratorio tenemos una disolución de alcohol etílico 96\% en masa, queremos obtener una disolución de 500ml a 0.1 M. Calcula cuántos mililitros necesitamos de la disolución original, y cuantos mililitros de agua necesitaremos. \\
Datos: densidad del alcohol etílico = \(789kg/m^{3}\).

\section{[2.5pt]}
\label{sec:org466d254}
Tenemos una sustancia S de una molaridad desconocida, pero sabemos que con 50ml de esta sustancia y enrasando con agua hasta los 380ml obtenemos la misma disolución a 0.3M. Cual es la molaridad de la sustancia inicial?

\section{[3pt]}
\label{sec:orgdb42b92}
Tenemos una disolución de ácido clorhídrico a 0.8 M. Para unos experimentos en el laboratorio necesitaremos esta disolución en distintas concentraciones. Nos piden cuantos mililitros de esta disolución necesitaremos para:
\begin{itemize}
\item Tener una mezcla de 100ml a 0.2 M
\item Tener una mezcla de 150ml a 0.1 M
\item Tener una mezcla de 200ml a 1M
\end{itemize}
Y justificar el procedimiento.

\section{[2pt]}
\label{sec:org95229f3}

\section{[0.5pt]}
\label{sec:orgcf73277}
Cuál es la condición para que la molalidad y la molaridad tengan el mismo valor en las unidades del sistema internacional?
\end{document}