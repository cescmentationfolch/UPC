% Created 2018-02-07 mié 18:02
% Intended LaTeX compiler: pdflatex
\documentclass[11pt]{article}
\usepackage[utf8]{inputenc}
\usepackage[T1]{fontenc}
\usepackage{graphicx}
\usepackage{grffile}
\usepackage{longtable}
\usepackage{wrapfig}
\usepackage{rotating}
\usepackage[normalem]{ulem}
\usepackage{amsmath}
\usepackage{textcomp}
\usepackage{amssymb}
\usepackage{capt-of}
\usepackage{hyperref}
\usepackage[margin=3cm]{geometry}
\usepackage{xfrac}
\usepackage{titling}

\setlength{\droptitle}{-2cm}
\author{Dean Zhu}
\date{9 de Febrero, 2018}
\title{Examen Química: Disoluciones
,,}

\begin{document}

\maketitle
\section{[2.5pt]}
\label{sec:org3540b65}
En el laboratorio tenemos una disolución de alcohol etílico 96\% en masa, queremos obtener una disolución de 500ml a 0.1 M. Calcula cuántos mililitros necesitamos de la disolución original, y cuántos mililitros de agua necesitaremos. \\
Datos: densidad del alcohol etílico = \(789kg/m^{3}\).

\section{[2.5pt]}
\label{sec:orgf175bfa}
Tenemos una sustancia S de una molaridad desconocida, pero sabemos que con 50ml de esta sustancia y enrasando con agua hasta los 380ml obtenemos la misma disolución a 0.3M. Cuál es la molaridad de la sustancia inicial?

\section{[3pt]}
\label{sec:org73262a9}
Tenemos una disolución de ácido clorhídrico a 0.8 M. Para unos experimentos en el laboratorio necesitaremos esta disolución en distintas concentraciones. Nos piden cuántos mililitros de esta disolución necesitaremos para:
\begin{itemize}
\item Tener una mezcla de 100ml a 0.2 M
\item Tener una mezcla de 150ml a 0.1 M
\item Tener una mezcla de 200ml a 1M
\end{itemize}
Justifica el procedimiento.

\section{[2pt]}
\label{sec:org96d5564}
Sabemos el magnesio reacciona con el nitrogeno.
\[ \text{M}_{s} + \text{N}_{2(g)} \rightarrow \text{Mg}_{3}\text{N}_{(s)} \]
\begin{itemize}
\item Iguala la ecuación
\item Cuántos gramos de nitruro de magnesio se formará si reaccionan 3,25g de magnesio
\end{itemize}

\section{[0.5pt]}
\label{sec:orgcc8700a}
Cuál es la condición para que el porcentaje en masa tenga el mismo valor que el porcentaje en volumen, suponed que la disolución se compone de un soluto y un solvente?
\end{document}