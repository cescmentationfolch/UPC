% Created 2018-02-06 mar 13:13
% Intended LaTeX compiler: pdflatex
\documentclass[11pt]{article}
\usepackage[utf8]{inputenc}
\usepackage[T1]{fontenc}
\usepackage{graphicx}
\usepackage{grffile}
\usepackage{longtable}
\usepackage{wrapfig}
\usepackage{rotating}
\usepackage[normalem]{ulem}
\usepackage{amsmath}
\usepackage{textcomp}
\usepackage{amssymb}
\usepackage{capt-of}
\usepackage{hyperref}
\usepackage[margin=3cm]{geometry}
\usepackage{xfrac}
\author{Dean}
\date{\today}
\title{General Motion}
\hypersetup{
 pdfauthor={Dean},
 pdftitle={General Motion},
 pdfkeywords={},
 pdfsubject={},
 pdfcreator={Emacs 25.3.1 (Org mode 9.1.3)},
 pdflang={English}}
\begin{document}

\maketitle
\section{Examen Quimica}
\label{sec:org0503a09}
\subsection{[3pt] 1}
\label{sec:org875f275}
Tenemos una disolución de ácido clorhídrico a 0.5 M. Para unos experimentos necesitaremos esta disolución en distintas concentraciones. Cuantos mililitros necesitaremos para:
\begin{enumerate}
\item Tener una mezcla de 100ml a 0.2 M
\item Tener una mezcla de 150ml a 0.1 M
\item Tener una mezcla de 200ml a 1M
\end{enumerate}
\end{document}