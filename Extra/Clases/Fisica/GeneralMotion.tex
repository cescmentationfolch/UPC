% Created 2018-02-03 Sat 13:41
% Intended LaTeX compiler: pdflatex
\documentclass[11pt]{article}
\usepackage[utf8]{inputenc}
\usepackage[T1]{fontenc}
\usepackage{graphicx}
\usepackage{grffile}
\usepackage{longtable}
\usepackage{wrapfig}
\usepackage{rotating}
\usepackage[normalem]{ulem}
\usepackage{amsmath}
\usepackage{textcomp}
\usepackage{amssymb}
\usepackage{capt-of}
\usepackage{hyperref}
\usepackage[margin=3cm]{geometry}
\usepackage{xfrac}
\date{\today}
\title{General Motion}
\hypersetup{
 pdfauthor={},
 pdftitle={General Motion},
 pdfkeywords={},
 pdfsubject={},
 pdfcreator={Emacs 25.3.1 (Org mode 9.1.3)}, 
 pdflang={English}}
\begin{document}

\maketitle

\section{General Motion}
\label{sec:orga83b068}
El objetivo de este tema es describir el movimiento de un objeto que no necesariamente es uniformemente acelerado. El caso mas común en la realidad.
\subsection{Velocidad y aceleración}
\label{sec:orgfa6f77a}
La velocidad por definición nos dice cuántos metros recorremos en un cierto tiempo, i.e. Un cambio de posicion respecto a un cambio de tiempo. Por tanto si tenemos un objeto que se mueve en MRU la velocidad es simplemente \(\frac{x_{1} - x_{0}}{t_{1} - t_{0}}\), en cambio cuando la función que describe el movimiento se complica es mucho más dificil describir la velocidad, ya que va variando según el tramo. En MRUA la velocidad pasa a ser \(v(t) = v_{0} + at\) una función que depende del tiempo.  

Pero en un caso más general no tiene porque ser cierto, asimismo podemos coger una aproximación. 
\[ v_{t_{0} \rightarrow t_{1}} = \frac{x_{1} - x_{0}}{t_{1} - t_{0}} \]

Tambien nos damos cuenta de que a medida que cogemos una \(t_{1}\) más cercana a \(t_{0}\), la aproximación pasa a ser mejor. Cuando cogemos un punto muy cercano, justamente obtenemos la \textbf{\textbf{\emph{derivada}}} de la función

\subsubsection{Derivada}
\label{sec:orga9f3b01}
La derivada de \(x(t)\) respecto el tiempo, (escrita como \(x'(t)\) ó \(\frac{dx}{dt}\)) simplemente es una operación que nos indica el cambio de la función en un instante. Por tanto podemos definir:
\[ 
v(t) =  \frac{dx}{dt}
\]
Ya que como hemos visto antes, la velocidad simplemente es el cambio de posición en el tiempo.
Análogamente la aceleración es el cambio de posición en un instante y por tanto 
\[
a(t) = \frac{dv}{dt} = \frac{d^{2}x}{dt^{2}}
\]
Si has podido seguir hasta este punto, ya deberías entender la relación entre \(x(t), v(t)\) y \(a(t)\)

\begin{enumerate}
\item Ejemplos
\label{sec:org0dfbc07}

   La derivada de \(ax^{b}\) es simplemente multiplicar por el exponente y restarle uno \(\implies b \cdot ax^{b-1}\)
   La derivada de una constante C es simplemente 0. \\
   Para derivar un polinomio \(p(x) = 3x^{3} + 5x^{2} + 3x + 5\) Nos basta con derivar cada parte por separado, entonces 
   \[ 
   \frac{dp}{dx} = 3 \cdot3x^{3-1} + 2 \cdot 5x^{2-1} + 1 \cdot 3x^{1-1} + 0 = 9x^{2} + 10x + 3 
   \]
\end{enumerate}
\subsubsection{Integral}
\label{sec:orgc200a45}
Puede existir el problema inverso, conocemos la función de la velocidad \(v(t)\)pero no la de posición \(x(t)\). Sabemos por el apartado anterior que:
\[
v(t) =  \frac{dx}{dt}
\]
Necesitariamos una operación que desiciese la derivada, esa es justamente la Integral \(\int\). Obtendríamos por tanto:
\[
\int v(t) dt = x(t)
\]
Análogamente el caso anterior: 
\[
\int a(t) dt = v(t)
\]
\[
\iint a(t) dt^{2} = x(t)
\]

\begin{enumerate}
\item Ejemplos
\label{sec:orgcf6f794}

Como integrar es lo opuesto a derivar, haremos lo contrario a la derivada.
Si los pasos para derivar un polinomio son multiplicar y restar 1 al exponente haremos lo contrario. sumaremos uno al exponente y divideremos.
Por tanto la integral de \(ax^b\) es \(\frac{ax^{b+1}}{b+1}\) y la integral de un número N es Nx.\\
Pero nos tenemos que dar cuenta que si la derivada de una constante es 0, está desaparece de la ecuación por tanto siempre es posible que tenga una constante de más, por tanto a todas las integrales le sumaremos C. Esto se verá más claro en el ejemplo siguiente. \\
Para integrar un polinomio \(q(x) = 9x^{2} + 10x + 3\), integramos cada parte por separado. Si realmente la integral deshace la derivada deberíamos obtener  \(p(x) = 3x^{3} + 5x^{2} + 3x + 5\)
   \[ 
   \int q dx  =  \int 9x^{2} + 10x + 3 dx = \frac{9x^{2+1}}{2+1} + \frac{10x^{1+1}}{1+1} + 3x = 3x^{3} + 5x^{2} + 3x
   \]
Fijemonos que no obtenemos el polinomio original, ya que la derivada de 5 es 0 y desaparece. de hecho cualquier polinomio \(p_{c}(x) = 3x^{3} + 5x^{2} + 3x + C\), tiene derivada igual a q(x). Por tanto para diferenciarlas necesitamos sumar esa C.
\end{enumerate}


\subsection{Caso particular MRUA.}
\label{sec:org7956e0e}
Si nos fijamos en el caso del MRUA, podemos coger la equación de posición y considerar sus derivadas

\begin{alignat*}{6}
                x(t)& = x_0& +& v_{0}t& +& \frac{1}{2}at^{2}& \\
\frac{dx}{dt} = v(t)& =   0& +& v_{0}&  +& at& \\
\frac{dv}{dt} = a(t)& =   0& +& 0&      +& a&
\end{alignat*}
Y obtenemos exactamente las tres ecuaciones del movimiento, Si integramos a partir de la ecuación de la aceleración obtenemos lo mismo: 
\begin{alignat*}{4}
    a(t)& = a& \\
    \int a(t) dt = v(t)& = at& +& C& \\
    \int v(t) dt = x(t)& = \frac{1}{2}at^{2}& +& Ct& + D
\end{alignat*}
Vemos aquí otra vez la necesidad de sumar constante ya que sino no podemos obtener las ecuacion originales.
Para saber cual es la constante necesaria simplemente consideramos un punto de la función donde conozcamos el valor: 
\[
v(0) = a\cdot(0) + C = v_{0} \implies C = v_{0}
\]
\[
x(0) = \frac{1}{2}a(0)^{2}& +& v_{0}(0)& + D = x_0 \implies x_0 = D
\]
Por tanto
\[
a(t) = a
\]
\[
v(t) = v_{0} + at
\]
\[
x(t) = x_{0} + v_{0}t + \frac{1}{2}at^{2}
\]
\end{document}