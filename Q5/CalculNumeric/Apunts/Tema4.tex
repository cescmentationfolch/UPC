% Created 2018-01-10 Wed 16:00
% Intended LaTeX compiler: pdflatex
\documentclass[11pt]{article}
\usepackage[utf8]{inputenc}
\usepackage[T1]{fontenc}
\usepackage{graphicx}
\usepackage{grffile}
\usepackage{longtable}
\usepackage{wrapfig}
\usepackage{rotating}
\usepackage[normalem]{ulem}
\usepackage{amsmath}
\usepackage{textcomp}
\usepackage{amssymb}
\usepackage{capt-of}
\usepackage{hyperref}
\date{\today}
\title{}
\hypersetup{
 pdfauthor={},
 pdftitle={},
 pdfkeywords={},
 pdfsubject={},
 pdfcreator={Emacs 25.3.1 (Org mode 9.1.3)}, 
 pdflang={English}}
\begin{document}

\tableofcontents

\section{Integracio Numerica}
\label{sec:org9c6a943}

\subsection{Metodes}
\label{sec:org44c12d5}

\subsubsection{Simple o Composta}
\label{sec:orge1d3555}

\subsubsection{Segon les dades}
\label{sec:org46af586}
\begin{enumerate}
\item Newton-Cotes: els punts estan fixats
\label{sec:org1dcde48}
\item Gauss: Podem avaluar la funcio en un punt
\label{sec:org707793b}
\end{enumerate}
\subsubsection{Oberta o tancada}
\label{sec:org7fd74db}

\subsection{Ordre d'una quadratura}
\label{sec:org19f78b7}
Una quadratura es d'ordre q, si integra exactament un polinomi de grau q. (per exemple si l'error depen de la derivada q+1-essima)

\subsection{Quadratura: Newton-Cotes}
\label{sec:org10b6174}
Fixats els punts \(\{x_{i} \}_{i=0\div n}\), volem una formula
\[ I = \int_{a}^{b} f(x) dx \approx \sum_{i=0}^{n} w_{i}f(x_i) \]


\subsubsection{Formula de trapezi}
\label{sec:org2b7ac8b}
\[ I = w_{0}f(a) + w_{1}f(b) + E_{1} \] 
on \(w_{0} = w_{1} = \frac{h}{2}, E_{1} = -\frac{h^3}{12}f^{''}(\mu), \mu \in [a,b].\)

\subsubsection{Quadratura de Simpson}
\label{sec:org83af6a9}
\[ I = w_{0}f(x_{0}) + w_{1}f(x_{1}) + w_{2}f(x_{2}) + E_2 \]
on \(w_{0} = w_{2} = \frac{1}{3}, w_{1} = \frac{4}{3}, E_2 = - \frac{(b-a)^3}{12m^2}f^{4)}(\mu) , \mu \in [a,b].\)

\section{Metodes numerics per EDOs}
\label{sec:orgb619b82}
Una EDO de ordre N te forma \(\frac{d^{n}y}{dx^{n}} = f(x, y, \frac{dy}{dx}, .. , \frac{d^{n-1}y}{dx^{n-1}, a < x < b\)
\end{document}