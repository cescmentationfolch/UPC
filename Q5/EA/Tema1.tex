% Created 2017-12-11 Mon 14:06
% Intended LaTeX compiler: pdflatex
\documentclass[11pt]{article}
\usepackage[utf8]{inputenc}
\usepackage[T1]{fontenc}
\usepackage{graphicx}
\usepackage{grffile}
\usepackage{longtable}
\usepackage{wrapfig}
\usepackage{rotating}
\usepackage[normalem]{ulem}
\usepackage{amsmath}
\usepackage{textcomp}
\usepackage{amssymb}
\usepackage{capt-of}
\usepackage{hyperref}
\usepackage[margin=3cm]{geometry}
\usepackage{xfrac}
\date{\today}
\title{Estructures Algebraiques: Tema 1}
\hypersetup{
 pdfauthor={},
 pdftitle={Estructures Algebraiques: Tema 1},
 pdfkeywords={},
 pdfsubject={},
 pdfcreator={Emacs 25.3.1 (Org mode 9.1.3)}, 
 pdflang={English}}
\begin{document}

\maketitle
\setcounter{tocdepth}{4}
\tableofcontents


\section{{\bfseries\sffamily TODO} Grup}
\label{sec:org03bf409}
\section{{\bfseries\sffamily TODO} Interseccio, unio, producte i generacio}
\label{sec:org0ee59dc}
\section{{\bfseries\sffamily TODO} Ordre d'un element, grup ciclic}
\label{sec:org267a00f}
\section{{\bfseries\sffamily TODO} Morfismes de grups}
\label{sec:org03a7a39}

\section{Classes laterals}
\label{sec:orgef4e261}
\subsection{Definicio}
\label{sec:org291f375}
Sigui G un grup i H in subgrup de G. \(a,b \in G\). \\
Definim \(a \sim b (esquerra) \iff a^{-1}b \in H\)
\subsection{Definicio de la classe d'equivalencia}
\label{sec:org1f64874}
\(\bar{a}\) := \{ \(b \in G\) | \(a \sim b\)\} \\
\(aH\) := \{ \(ax\) | \(x \in H\)\} \\
Observem que aH = \(\bar{a}\) \\
A més H i aH tenen el mateix cardinal.

\subsection{Definicio Classe Lateral:}
\label{sec:org8d6b397}
Anomenem \(G/H = \{aH | a \in G\}\) el conjunt de les classes laterals per l'esquerra (conjunt quocient). \\
Diem també que l'índex de H en G es \([G:H] = |G/H|\) = nombre d' elements de G modul H

\subsubsection{Observacio:}
\label{sec:orgeb17027}
\begin{enumerate}
\item El nombre de classes laterals per l'esquerra es el mateix que el nombre de classes laterals per la dreta.
\item L'índex és multiplicatiu. K \(\subset\) H \(\subset\) G. Llavors [G:K] = [G:H] * [H:K]
\end{enumerate}

\subsection{Teorema de Lagrange}
\label{sec:orgc82c73d}
Sigui G un grup i H un subgrup, G finit. \\
Aleshores \(|G| = |G/H| * |H| \iff |G/H| = \frac{|G|}{|H|}\), i a més |H| divideix |G|.

\subsubsection{{\bfseries\sffamily TODO} Demo}
\label{sec:org01f10b5}

\section{Subgrup normal, Grup quocient}
\label{sec:orgfef0244}
\subsection{Definicio: Subgrup Normal}
\label{sec:org27bc38b}
Sigui G un grup, H un subgrup de G. H és subgrup normal de G si aH = Ha \(\forall\) a \(\in\) G.
\subsection{Lema}
\label{sec:orga570542}
f: \(G_1 \to G_2\) morfime de grups. Aleshores,
\begin{enumerate}
\item Si \(H_1 \vartriangleleft G1 \implies f(H_1) \vartriangleleft f(G_1)\).
\item Si \(H_2 \vartriangleleft G2 \implies f^{-1}(H_2) \vartriangleleft G_1\).
\end{enumerate}
\subsubsection{{\bfseries\sffamily TODO} Demo}
\label{sec:org01ea247}
\subsection{Observacio}
\label{sec:org45bdedb}
\(H \subseteq K \subseteq G\), H,K subgrups de G. Aleshores. \\
Si \(H \vartriangleleft G \implies H \vartriangleleft K\). El reciproc es fals.
\subsection{Observacio}
\label{sec:org18f28eb}
si \(H \vartriangleleft G\), aleshores (aH)*(bH) = (ab)H
\subsection{Definicio: Operacio interna}
\label{sec:org650e7ae}
Sigui \(H \vartriangleleft G\) i sigui G/H = \{aH | a \(\in\) G\} el conjunt de classes laterals per l'esquerra modul H. En G/H definim l'operacio interna:
\begin{alignat*}{5}
G/H &\times G/H &\to&\hspace{2pt}  G/H & \\
aH &\times bH &\mapsto&  (ab)H &
\end{alignat*}
\subsection{Corol.lari}
\label{sec:orgc83a52a}
\(G/H\) es un grup i s'anomena el grup quocient de G per H.
\subsection{Exercici: La aplicacio quocient es un morfisme}
\label{sec:org4082f08}

\section{Primer teorema d'isomorfisme}
\label{sec:org244d396}

\subsection{{\bfseries\sffamily TODO} Teorema:}
\label{sec:orgce2857f}
Sigui \(f: G_1 \to G_2\) morfisme de grups, Sigui \(H \vartriangleleft G_1\), i sigui l'aplicació

\begin{alignat*}{2}
\tilde{f}: G_1/H &\to G_2 \\
aH &\mapsto \tilde{f}(aH) := f(a)
\end{alignat*}
\begin{figure}[htbp]
\centering
\includegraphics[width=.9\linewidth]{./images/primeriso.jpg}
\caption{\label{fig:org93aa0a9}
Primer teorema d'isomorfisme}
\end{figure}

\section{El grup multiplicatiu d'un cos finit}
\label{sec:org99ee81a}

\subsection{Definicio}
\label{sec:orga2e033c}
Sigui \(\mathbb{K}\) un cos. El grup multiplicatiu de \(\mathbb{K}\) és \\
\begin{equation*}
\mathbb{K}^* = \mathbb{K} \setminus \{0\} = \{x \in \mathbb{K} \mid x \neq 0\}
\end{equation*}

\subsection{Teorema}
\label{sec:orga12f918}
Sigui \(\mathbb{K}\) un cos. Sigui G un subgrup finit de \(\mathbb{K}^*\). Aleshores G és cíclic

\subsubsection{{\bfseries\sffamily TODO} demo}
\label{sec:org953cc86}

\section{Grup simples}
\label{sec:orgb5ea09b}
\subsection{Definicio}
\label{sec:orgc85ae6c}
Sigui G un grup no trivial. Direm que G es simple si els unics subgrups normals de G son \{1\} i G.
\subsection{Proposicio}
\label{sec:org3a2459f}
Sigui G un grup no trivial. Son equivalents
\begin{enumerate}
\item G es simple i abelia
\item |G| = p, on p es primer
\item \(G \cong \mathbb{Z}/p\mathbb{Z}\)
\end{enumerate}

\subsubsection{{\bfseries\sffamily TODO} Demo}
\label{sec:org8baf1a8}
\subsection{Teorema de Feit-Thompspn}
\label{sec:orgfa76628}
Sigui G grup simple, Suposem |G| es senar. Aleshores G es ciclic i \(G \cong \mathbb{Z}/p\mathbb{Z}\).
\subsection{Teorema}
\label{sec:org076f5ea}
Sigui n \(\geq\) 5, Aleshores \(\mathcal{A}_n\) es simple
\subsubsection{{\bfseries\sffamily TODO} Demo}
\label{sec:org0c6d124}
\subsection{Proposicio}
\label{sec:org8d22086}
    Sigui G un grup, \(H \vartriangleleft G\). Aleshores,\\
G/H es grup simple \(\iff\) H es un element maximal en el conjunt \{K | \(K \vartriangleleft G\), \(K \neq G\)\}
\subsubsection{{\bfseries\sffamily TODO} Demo}
\label{sec:orgf6983d9}

\section{Grup resolubles}
\label{sec:org8c544cf}

\subsection{Definicio torre normal}
\label{sec:orgdb17b2c}
Una torre normal de G es \(G = G_0 \vartriangleright G_1 \vartriangleright G_2 \vartriangleright \ldots \vartriangleright G_n = \{1\}\) on G es un grup i \(G_i \vartriangleleft G_{i+1}\). \\
Anomenem n la \emph{longitud de la torre} \\
\(G_{i-1}/G_i\) s'anomenen els \emph{quocients de la torre} \\

A mes definim:
\begin{itemize}
\item \textbf{Torre normal abeliana}: Torre normal amb quocients abelians.
\item \textbf{Torre normal simple/serie de composicio}: Torre normal amb quocients abelians
\end{itemize}

\subsection{Definicio Grup Resoluble}
\label{sec:org553b19d}
Direm que G es resoluble si te una torre normal abeliana.

\subsection{Teorema: Segon Teorema d'isomorfisme}
\label{sec:org81ca656}
Sigui G grup i H,K dos subgrups de G. Suposem \(H \vartriangleleft G\). Aleshores:
\begin{enumerate}
\item \(H \cap K \vartriangleleft K\)
\item \(H \cdot K\) es subgrup de G
\item \(H \vartriangleleft H \cdot K\)
\item A mes a mes, \(\sfrac{K}{H \cap K} \cong \sfrac{H \cdot K}{H}\)
\end{enumerate}

\subsubsection{{\bfseries\sffamily TODO} Demo}
\label{sec:org959dffc}

\subsection{Teorema: Jordan-Holder}
\label{sec:org85b1248}
\begin{displaymath}
    \text{Sigui G un grup i}
               \left\{\begin{array}{ll}
G = G_0 \vartriangleright G_1 \vartriangleright G_2 \vartriangleright \ldots \vartriangleright G_n = \{1\} \\
G = H_0 \vartriangleright H_1 \vartriangleright H_2 \vartriangleright \ldots \vartriangleright H_m = \{1\}
                \end{array}
\right\rbrace
              \text{Dues series de composicio de G}
\end{displaymath}

Aleshores n = m, i \(\exists \sigma \in \mathcal{S}_n\) tal que \(\sfrac{H_i}{H_{i+1}} \cong \sfrac{G_{\sigma(i)}}{G_{\sigma(i)+1}}\).

\subsubsection{{\bfseries\sffamily TODO} Demo}
\label{sec:orgd877252}

\subsection{Proposicio}
\label{sec:orga5c6a63}
Sigui G un grup, H un subgrup de G. Aleshores
\begin{enumerate}
\item Si G es resoluble \(\implies\) H es resoluble
\item Si \(H \vartriangleleft G\) i G es resoluble \(\implies \sfrac{G}{H}\) es resoluble
\item Si \(H \vartriangleleft G\) i H i \(\sfrac{G}{H}\) son resolubles \(\implies\) G es resoluble
\end{enumerate}

\section{Accio d'un grup en un conjunt}
\label{sec:orgf9bbb9b}
\subsection{Definicio: Accio d'un grup en un conjunt}
\label{sec:orgb00c1c5}
Sigui G un grup. SIgui X un conjunt. Una accio de G en X es una aplicacio
\begin{alignat*}{2}
\varphi : G \times X &\to X \\
(a, x) &\mapsto \varphi(a,x) = ax
\end{alignat*}
tal que:
\begin{enumerate}
\item \(a \cdot (b \cdot x) = (a \cdot b) \cdot x \hspace{10pt}  \forall a,b \in G, \forall x \in X\)
\item \(1 \cdot x = x \hspace{10pt} \forall x \in X\)
\end{enumerate}
\subsection{Observacio}
\label{sec:orgecb97fe}
Hi ha una bijeccio entre \\
\{\(\varphi: G \times X \to X \mid \varphi \text{ accio de G en X}\)\} \(\leftrightarrow\) \{\(\phi: G \to Perm(X) \mid \phi \text{ morfisme de grups}\)\}
\subsection{Definicio: Orbita d'un element}
\label{sec:orgc82e410}
L'orbita de \(x \in X\) es el subconjunt \(G \cdot x = \{ax \mid a \in G \} \subseteq X\)
\subsection{Definicio: L'estabilitzador/grup d'isotropia d'x \(\in\) X}
\label{sec:org3febc0f}
Gx := \{\(a \in G \mid ax = x \} \subseteq G\), es un subgrup de G.
\subsection{Lema:}
\label{sec:org0f6a74d}
Si x,y estan en la mateixa orbita, els seus estabilitzadors son conjugats. \\
Concretament, si y = ax \(\implies G_y = aG_{x}a^{-1}\)
\subsubsection{{\bfseries\sffamily TODO} DEMO}
\label{sec:org0b19f0b}

\subsection{Proposicio}
\label{sec:orga38ccb3}
L'aplicacio 
\begin{alignat*}{3}
G \cdot x &\to \sfrac{G}{G_x}& \\
ax &\mapsto a\cdot G_x&
\end{alignat*}
esta ben definida i es bijectiva. En particular, \\
\begin{enumerate}
\item \(\vert G \cdot x \vert = \vert \sfrac{G}{G_x}\vert = [G:G_x]\)
\item Si G es finit, \(\vert G \cdot x \vert \text{ divideix } \vert G \vert\)
\item Si X es finit, \(\vert X \vert = \sum_{i=1}^{n} \vert G \cdot x_i \vert = \sum_{i=1}^n [G:G_{x_i}]\)
\end{enumerate}
\subsubsection{{\bfseries\sffamily TODO} DEMO}
\label{sec:org2c9edef}

\subsection{Definicio: punt fix}
\label{sec:orgebce142}
\(x \in X\) es un punt fix per l'accio si ax = x \(\forall a \in G\). En particular\\
\(G \cdot x = \{ax \mid a \in G\} = \{x\}\), \(G_x = \{ a \in G \mid ax = x \} = G\)

\subsection{Definicio: Accio Transitiva}
\label{sec:orgcf52ffb}
\(G \times X \to X\) es accio transitiva si \(\forall\) x,y \(\in\) X, \(\exists\) a \(\in\) G \text{ tal que } y = ax. \\
En aquest cas. G \(\cdot\) y = X \(\forall\) \quad y \(\in\) X.

\subsection{Definicio: Accio Fidel}
\label{sec:org55c0ca4}
\(G \times X \to X\) es accio fidel si \(\forall\) a \(\neq\) b, a,b \(\in\) G. Aleshores m\(_{\text{a}}\) \(\neq\) m\(_{\text{b}}\), on
\begin{alignat*}{3}
m_a: &X &\to X \\
&x &\mapsto ax
\end{alignat*}
m\(_{\text{a}}\) \(\in\) Perm(x)

\subsubsection{Observacio:}
\label{sec:orgab1b845}
Si be \(G \times X \to X \cong m: G \to Perm(x)\) es morfisme de grups, si imposem que es fidel, el morfisme es injectiu. A mes si X es finit l'accio es isomorf a un subgrup del grup simetric.

\subsection{Accio per translacio en X, quan X = G}
\label{sec:orge6d9d26}
Sigui G un grup, definim
\begin{alignat*}{4}
&G \times &G &\to G \\
&a &x &\mapsto a \cdot x = ax
\end{alignat*}
I es efectivament una accio. 
\subsection{Teorema de Cayley}
\label{sec:org91dd616}
Sigui G un grup finit, n = |G|. Aleshores G es isomorf a un subgrup del grup simetric \(\mathcal{S}_n\)
\subsubsection{{\bfseries\sffamily TODO} Demo}
\label{sec:org2123c85}


\subsection{Definicio: Accio per conjugacio de G en X = G}
\label{sec:orgc0a3019}
\begin{alignat*}{4}
\label{eq:org26be6c1}
&G \times &G &\to G \\
&a &x &\mapsto a \cdot x = axa^{-1}
\end{alignat*}

\subsubsection{Proposicio:}
\label{sec:orgc06909c}

\(x \in G \text{ es punt fix } \iff a \cdot x = x \quad \forall a \in G \iff axa^{-1} = x \forall a \in G \iff ax = xa \quad \forall a \in G \iff x \in \mathcal{Z}(G) = \{ x \in G \mid ax = xa \quad \forall a \in G \} = \text{ centre de G}\). El centre de G es subgrup.

\subsubsection{Proposicio:}
\label{sec:orgbf426f0}
L'estabilitzador de y \(\in\) G es \(G_y = \{a \in G \mid a \cdot y = y \} = \{ a \in G \mid aya^{-1} = y \} = \{ a \in G \mid ay = ya \} = \mathcal{Z}_{G}(y)\),  centralitzador de G. El centralitzador tambe es un subgrup de G.

\subsection{Definicio: Accio per translacio en les classes laterals}
\label{sec:orgcebfe8d}
Sigui G grup, H subgrup de G i X = \(\sfrac{G}{H}\) = \{ aH \mid a \(\in\) G\}
\begin{alignat*}{4}
&G \times &\sfrac{G}{H} &\to \sfrac{G}{H} \\
&a &bH &\mapsto abH
\end{alignat*}

\begin{itemize}
\item Es una accio transitiva.
\item si \(aH \in X = \sfrac{G}{H} \text{: L'estabilitzador de aH es } G_{aH} = \{ b \in G \mid b(aH) = aH \} = aHa^{-1}\)
\end{itemize}


\subsection{Definicio: Accio per conjugacio en els subgrups}
\label{sec:org479222d}
Sigui G grup i \(X = \{ H \mid \text{ H subgrup de G} \}.\)
\begin{alignat*}{4}
&G \times &\text{\{sg. de yG\}} &\to \text{\{sg. de G\}, conjugat de H} \\
&a &H &\mapsto aHa^{-1}
\end{alignat*}
Si H es subgrup de G, l'orbita d'H es: \\
\(G\cdot H =\{a \cdot H\mid a\in G \} = \{aHa^{-1} \mid a \in G \} \text{: els conjugats de H}\) \\
H es punt fix per l'accio si \(a\dotH = H \iff aHa^{-1} = H \quad \forall a \in G \iff \text{H es subgrup normal de G}\) 
L'estabilitzador de H es: \(G_H = \{a \in G \mid a \cdot H\} = \{ a \in \G \mid aHa^{-1} = H \} = N_{G}(H)\): Normalitzador de H en G\\
Sabem que \(\vert G \cdot H\vert = [G : G_H ].\) Per tant. \\
\(H \vartriangleleft G \iff \text{H es punt fix per l'accio} \iff \text{L'orbita de H te un sol punt } \iff \vert G \cdot H \vert = 1 \iff [ G : G_H] = 1 \iff G_H = G \iff N_{G}(H) = G\) 

\subsection{Teorema de Cauchy}
\label{sec:org3b809a2}
Sigui G un grup finit, |G| = n. Sigui p primer tal que p|n. \\
ALeshores, \(\exists\) x \(\in\) G tal que ord(x) = p

\subsubsection{{\bfseries\sffamily TODO} Demo}
\label{sec:orgd774f9b}

\section{Subgrups de Sylow}
\label{sec:org45b6535}

\subsection{Definicio: p-grups: Subgrups de Sylow}
\label{sec:org66e33cb}
Sigui G un grup i p un nombre primer. Aleshores,  G es un p-grup \iff \(\vert{}\) G \(\vert{}\) = p\(^{\text{r}}\) per a algun r \(\ge\) 0.

\subsection{Teorema:}
\label{sec:org05b9428}
Sigui G un p-grup. Aleshores, |G| = p\(^{\text{r}}\), r \(\ge\) 0, i:
\begin{enumerate}
\item G no trivial \(\iff \mathcal{Z}(G)\) no trivial.
\item G es resoluble
\item si G es simple, aleshores G \(\cong \sfrac{\mathbb{Z}}{p\mathbb{Z}}\)
\end{enumerate}

\section{Teoremas de Sylow}
\label{sec:org183d4f5}
\end{document}