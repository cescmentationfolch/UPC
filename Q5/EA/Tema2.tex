% Created 2017-12-14 jue 16:09
% Intended LaTeX compiler: pdflatex
\documentclass[11pt]{article}
\usepackage[utf8]{inputenc}
\usepackage[T1]{fontenc}
\usepackage{graphicx}
\usepackage{grffile}
\usepackage{longtable}
\usepackage{wrapfig}
\usepackage{rotating}
\usepackage[normalem]{ulem}
\usepackage{amsmath}
\usepackage{textcomp}
\usepackage{amssymb}
\usepackage{capt-of}
\usepackage{hyperref}
\usepackage[margin=3cm]{geometry}
\usepackage{xfrac}
\author{Dean}
\date{\today}
\title{Estructures Algebraiques: Tema 2}
\hypersetup{
 pdfauthor={Dean},
 pdftitle={Estructures Algebraiques: Tema 2},
 pdfkeywords={},
 pdfsubject={},
 pdfcreator={Emacs 25.3.1 (Org mode 9.1.3)},
 pdflang={English}}
\begin{document}

\maketitle
\setcounter{tocdepth}{4}
\tableofcontents


\section{Anell}
\label{sec:orgd27d9dd}

\subsection{Definicio Anell}
\label{sec:orgc222efc}
Un \textbf{anell} es un conjunt A amb dues operacions internes, la suma (+) i el producte (\(\cdot\)), tals que: \\
\begin{enumerate}
\item (A, +) es grup abelia
\item (A, \(\cdot\)) verifica la propietat associativa.
\item Propietat distributiva del producte respecte la suma.
\end{enumerate}
L'anell es commutatiu si \(ab = ba \quad \forall a,b \in A\) \\
L'anell es unitari si existeix neutre pel producte.

\subsection{Propietats:}
\label{sec:orgb4af416}
\begin{enumerate}
\item a \(\cdot\) 0 = 0
\item 1 = 0 \(\iff\) A = 0
\item (-1) \(\cdot\) x = -x
\item Suposem (A,+, \(\cdot\)) anell commutatiu amb unitat aleshores la suma es commutativa. (S'imposa a la definicio per si l'anell no es unitari)
\end{enumerate}


\subsection{Definicio A\(^{\text{*}}\)}
\label{sec:orga1a6fab}
\(A^{*} = \{ a \in A \mid \exists b \in A \text{ tal que } ab = 1 \}\), elements invertibles de A. \\
Llavors A\(^{\text{*}}\) es el grup multiplicatiu i es un grup abelia.

\subsection{Definicio Cos}
\label{sec:orgc69b7c1}
Un \textbf{cos} es un anell A commutatiu amb unitat tal que A\(^{\text{*}}\) = A $\backslash$ \{0\}.

\subsection{Definicio Subanell}
\label{sec:org3f2850f}
Un \textbf{Subanell} de A es un subconjunt B \(\subseteq\) A tal que amb la suma, el producte i el deutre de A 1\(_{\text{A}}\), es un anell.

\section{Ideals}
\label{sec:orgd8a8e65}

\subsection{Definicio ideals}
\label{sec:orgc07226a}
Sigui A un anell commutatiu amb unitat. Un \textbf{ideal de A} es un subconjunt I de A tal que cumpleix:
\begin{enumerate}
\item 0 \(\in\) I
\item \(\forall x, y \in I, \quad x + y \in I\)
\item \(\forall x \in I, \forall a \in A, ax \in I\)
\end{enumerate}


\subsubsection{Observacio}
\label{sec:org0ee89db}
La condicio 1 equival a que I \(\neq \varnothing\).

\subsubsection{Ideal impropi}
\label{sec:org09a4233}
Direm que un Ideal I es impropi si\(I = \varnothing\).

\subsubsection{Ideal principal}
\label{sec:org3e3c72a}
Sigui A anell i x \(\in\) A. El conjunt de multiples de x = \(\{ ax \mid a \in A \}\) es un ideal de A. Se l'anomena \textbf{ideal principal}

\begin{enumerate}
\item Observacio
\label{sec:orgb5e4624}
Tot ideal de \(\mathcal{Z}\) es principal.
\end{enumerate}

\subsubsection{Proposicio}
\label{sec:org5ccdd46}
\(I = A \iff 1 \in I \iff I \cap A^* \neq \varnothing\).

\subsubsection{Proposicio}
\label{sec:orgd84559d}
Sigui A anell commutatiu amb unitat. Aleshores, A es cos \(\iff \{\text{ideals de A}\} = \{0,A\}\)

\subsection{Operacions amb ideals}
\label{sec:org95d1c92}

\subsubsection{Interseccio}
\label{sec:orgb9ecdde}
Sigui A anell, I, J ideals de A.
La interseccio de I i J es I \(\cap\) J = \(\{ x \in A \mid x \in I, x \in J\}\) \\
Si I, J son ideals de A \(\implies\) I \(\cap\) J es ideal de A.

\subsubsection{Unio}
\label{sec:org9bff67d}
Sigui A anell, I, J ideals de A.
La unio de I i J es I \(\cup\) J = \(\{ x \in A \mid x \in I \text{ o } x \in J\}\) \\
En general l'unio de ideals no es ideal.

\subsubsection{Suma}
\label{sec:org7031c6f}
Sigui A anell, I, J ideals de A.
La suma de I i J es I + J = \(\{ x + y \mid x \in I, y \in J\}\) \\
Si I, J son ideals de A \(\implies\) I + J es ideal de A. $\backslash$\ A mes a mes es el menor ideal que conte \(I \cup J\).

\subsubsection{Generador}
\label{sec:org92bda1f}
Sigui A un anell, x\(_{\text{1}}\), \dots{}, x\(_{\text{r}}\) \(\in\) A. Aleshores <x1,\dots{},x\(_{\text{r}}\)> = \(\{ a_{1}x_{1} + \dots + a_{r}x_{r} \mid a_i \in A \}\) es el menor ideal de A que conte  x\(_{\text{1}}\), \dots{}, x\(_{\text{r}}\)
\end{document}